%%%%%%%%%%%%%%%%%%%%%%%%%%%
% PREAMBOLO DEL DOCUMENTO %
%%%%%%%%%%%%%%%%%%%%%%%%%%%
\documentclass[a4paper,11pt,oneside,top=3cm,bottom=3cm,left=3.5cm,right=3.5cm,openright,reqno,table]{book}

% openany - fa iniziare i capitoli direttamente nella pagina successiva
% openright - fa iniziare i capitoli nella prima pagina destra disponibile 
% fleqn  - allinea le formule a sinistra anzichè centrarle
% leqno - dispone la numerazione delle formule sulla sinistra o destra
% reqno - dispone la numerazione delle formule sulla destra
%
\usepackage{packages}
% Per non appesantire troppo questo file
% quasi tutti i pacchetti usati sono salvati in packages.sty
%
\linespread{1.5}
% Per avere la parola BOZZA scritta su tutte le pagine

% funziona solo in modalità PS
% Invece per i PDF ho risolto così:
% pdftk tesi.pdf background bozza.pdf output tesi_bozza.pdf
%
%%%%%%%%%%%%%%%%%%%%%%%%%%%%%%%%%
%   DOCUMENTO VERO E PROPRIO    %
%%%%%%%%%%%%%%%%%%%%%%%%%%%%%%%%%
\begin{document}
% FRONTESPIZIO %
\begin{titlepage}
\changepage{}{}{}{-7.5 mm}{}{}{}{}{}
% parametri per cambiare le dimensioni di una singola pagina in ordine:
% {textheight}{textwidth}{evensidemargin}{oddsidemargin}{columnsep}
% {topmargin}{headheight}{headsep}{footskip}
% se voglio centrare la pagina devo mettere bindingoffset/2
% i primi 5 parametri posso usarli con \changetext


\begin{center}
\includegraphics [width=.15\columnwidth, angle=0]{unisa}\\ % height
\vspace{0.5cm}
{\LARGE \scshape Università degli Studi di Salerno}\\
\vspace{0.5cm}
{\Large Dipartimento di Informatica}\\
\vspace{0.1cm}
{\large Corso di Laurea Triennale in Informatica}\\
\vspace{1.5cm}
{\Large \scshape Tesi di Laurea} \\
\vspace{4cm}
{\Huge \bfseries TITOLO TESI} \\
\vspace{5cm}

\begin{minipage}[t]{7cm}
\flushleft
\textsc{Relatore}

Prof. \textbf{Andrea De Lucia} \\
{\small Università degli studi di Salerno} \\[0.25cm]
\end{minipage}
\hfill
\begin{minipage}[t]{7cm}
\flushright
\textsc{Candidato}

\textbf{Nome Cognome} \\
Matricola: 0123456789
\end{minipage}

\vspace{3cm}

%& & \\
%& Candidato & \\
%& \textbf{Fabiano Pecorelli} & \\
{\small Anno Accademico YYYY-YYYY} %\\
%
%
\begin{comment}
\begin{table}[!h]
\centering
\begin{tabular}{c c c} %p{5cm}c
& Tesi di laurea & \\
& \textbf{Fabiano Pecorelli} & \\
& & \\[0.25cm]
Relatore \\
prof. \textbf{Andrea De Lucia} \\
{\small Università degli studi di Salerno} & & {\small Provincia di Salerno}\\
& & \\[0.5cm]
& {\small A.A. 2015-2016} & \\
\end{tabular}
\end{table}
\end{comment}
%
%
\end{center}

\end{titlepage}
%

\frontmatter
% quello che segue è in numerazione romana e i capitoli non verranno numerati
% se non si vuole che compaia il numero di pagina basta usare il comando:
%\nonumber

% RINGRAZIAMENTI %
\begin{titlepage}

\nonumber
\null \vspace {\stretch{1}}
	\begin{flushright}
%	\begin{verse}
\textit{INSERIRE QUI UNA DEDICA O UNA CITAZIONE} \\[5mm]
%	\end{verse}
	\end{flushright}



\end{titlepage}
% SOMMARIO %
\cleardoublepage
%\selectlanguage{italian}
\begin{abstract}

INSERIRE ABSTRACT
\\[1cm]
\end{abstract} 
% INDICI %
\phantomsection
\addcontentsline{toc}{chapter}{Indice}
\tableofcontents
% Il simbolo * serve per evitare che comapaia nell'indice
\clearpage
%\listoffigures
%\clearpage
%\listoftables
% GLOSSARIO
%\cleardoublepage
\phantomsection
\addcontentsline{toc}{chapter}{Glossario}
% per inserire l'elenco dei simboli e degli acronimi nell'indice
\printglossary
% Per stampare il glossario
% per aggiornarlo si deve eseguire da terminale:
% makeindex -s myDoc.ist -t myDoc.alg -o myDoc.acr myDoc.acn
% per inserire una voce nell'elenco:
% \newglossaryentry{voce_etichetta}{name={voce}, description={descrizione}}
% se non compare direttamente nel testo va inizializzata con:
% \glsadd{voce_etichetta}
% oppure se viene richiamata all'interno del testo:
% \gls{voce_etichetta}
% SIMBOLI E NOTAZIONI %
\cleardoublepage
\phantomsection
\addcontentsline{toc}{chapter}{Elenco delle figure}
% per inserire l'elenco dei simboli e degli acronimi nell'indice
%\printglossary[type=\acronymtype,title=Elenco delle figure]
% Per stampare l'elenco dei simboli
\listoffigures
\cleardoublepage
\phantomsection
\addcontentsline{toc}{chapter}{Elenco delle tabelle}
% per inserire l'elenco dei simboli e degli acronimi nell'indice
%\printglossary[type=\acronymtype,title=Elenco delle figure]
% Per stampare l'elenco dei simboli
\listoftables
% per aggiornarlo si deve eseguire da terminale:
% makeindex -s myDoc.ist -t myDoc.glg -o myDoc.gls myDoc.glo
% per inserire una voce nell'elenco:
% \newglossaryentry{voce_etichetta}{name={voce}, description={descrizione}}
% se non compare direttamente nel testo va inizializzata con:
% \glsadd{voce_etichetta}
% oppure se viene richiamata all'interno del testo:
% \gls{voce_etichetta}

\mainmatter
% quello che segue sarà in numerazione araba e i capitoli verranno numerati
%\part{Studio iniziale}
% CAPITOLI
\phantomsection
%\addcontentsline{toc}{chapter}{Introduzione}
\chapter{Introduzione}
\markboth{Introduzione}{}
% [titolo ridotto se non ci dovesse stare] {titolo completo}

\section{Motivazioni e Obiettivi} %\label{1sec:scopo}

\section{Risultati}

\section{Struttura della tesi}

\chapter{Stato dell'arte} %\label{1cap:spinta_laterale}
% [titolo ridotto se non ci dovesse stare] {titolo completo}
%

\begin{citazione}
Questo capitolo illustra lo stato dell'arte e i lavori presenti in letteratura sugli aspetti di ricerca trattati nel nostro studio. ECC ECC...
\end{citazione}

\newpage
\include{capitoli/design}
\chapter{Conclusioni} %\label{1cap:spinta_laterale}
% [titolo ridotto se non ci dovesse stare] {titolo completo}
%


\begin{citazione}
	BREVE SPIEGAZIONE CONTENUTO CAPITOLO
\end{citazione}

\newpage

%\part{Impatto ambientale}

\backmatter
\phantomsection
\chapter{Ringraziamenti}
\markboth{Ringraziamenti}{}
% [titolo ridotto se non ci dovesse stare] {titolo completo}
INSERIRE RINGRAZIAMENTI QUI 
\end{document}
